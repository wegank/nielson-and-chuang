\documentclass[../main.tex]{subfiles}
\begin{document}

\setcounter{chapter}{5}
\setcounter{section}{2}
\setcounter{exercise}{9}
\section{Applications: order-finding and factoring}
\subsection{Application: order-finding}

\begin{exercise}
\end{exercise}
\begin{proof}
Trivial.
\end{proof}

\bigskip
\begin{exercise}
\end{exercise}
\begin{proof}
By Lagrange's theorem, we have $r=\lvert x\rvert\leq\lvert\mathbb{Z}_N^*\rvert\leq N$.
\end{proof}
\begin{proof}[Proof (Elemental)]
Consider the set $E=\bigcup_{r=0}^N\{x^r\}$. Since $\lvert E\rvert=N+1>N$, there exist two integers $0\leq\alpha<\beta\leq N$ such that $x^\alpha\equiv x^\beta\Mod{N}$. We have $r\leq\beta-\alpha\leq N$.
\end{proof}

\bigskip
\begin{exercise}
\end{exercise}
\begin{proof}
For all $(y,y^\prime)\in(\{0,1\}^L)^2$, we have
\[
    \braket{x y^\prime\Mod{n}\vert x y\Mod{n}}=\braket{y^\prime\Mod{n}\vert y\Mod{n}}=\braket{y^\prime\vert y}=\delta_{y y^\prime}
\]

So $U$ is unitary.
\end{proof}

\bigskip
\begin{exercise}
\end{exercise}
\begin{proof}
We have
\begin{align*}
    \frac{1}{\sqrt{r}}\sum_{s=0}^{r-1}e^{\frac{2\pi i s k}{r}}\ket{u_s}
    &= \frac{1}{r}\sum_{s=0}^{r-1}\sum_{k^\prime=0}^{r-1}e^{\frac{2\pi i(k-k^\prime)s}{r}}\ket{x^k\Mod{N}} \\
    &= \frac{1}{r}\sum_{k^\prime=0}^{r-1}r\delta_{k k^\prime}\ket{x^k\Mod{N}} = \ket{x^k\Mod{N}}
\end{align*}

and
\[
    \frac{1}{\sqrt{r}}e^{-\frac{2\pi i s k}{r}}\ket{u_s}=\ket{x^0\Mod{N}}=\ket{1}
\]
\end{proof}

\bigskip
\begin{exercise}
\end{exercise}
\begin{proof}
We observe that
\[
    \ket{\varphi}=\sum_{j=0}^{2^t-1}\ket{j}\ket{x^j\Mod{N}}=\sum_{j=0}^{2^t-1}V\ket{j}\ket{0}
\]

In addition, we have
\begin{itemize}
    \item $\ket{j}\mapsto\ket{x^j\Mod{N}}$ is in $O(L^3)$. (Box 5.2)
    \item $\ket{x^j\Mod{N}}\ket{k}\mapsto\ket{k+x^j\Mod{N}}\ket{k}$ is in $O(L)$. (3.2.5)
\end{itemize}

So that $V$ is in $O(L^3)$.
\end{proof}

\bigskip
\begin{exercise}
\end{exercise}
\begin{enumerate}
    \item \begin{proof}
    The first statement is trivial under prime factorization.
    \end{proof}
    \item \begin{proof}
    We show that $\gcd$ is in $O(L^2)$. 
    
    Following the proof on Page 629, we note as $k_i=\lfloor\log_2 r_i\rfloor+1$ number of digits of $r_i$. Since $(r_i)_i$ is decreasing and $r_{i+1}\leq \frac{r_i}{2}$, for every eligible $i$, we have
    \[
        k_{i+1}\leq k_i
    \]
    \[
        k_{i+2}\leq k_i-1
    \]
    
    Since a schoolbook long division of $m$-bit by $n$-bit requires at most $(m-n+1)n+O(\max(m,n))$ binary operations, the number of operations in the Euclid's algorithm is bounded by
    \[
        \sum_{i}(k_{i-1}-k_i+1)k_i\leq\sum_i(k_{i-2}-k_{i+2})k_i\leq 4k_0^2\in O(L^2)
    \]
    with its remainder
    \[
        O(\sum_{i}\max(k_{i-1},k_i))\subset O(\sum_i L)=O(L\log(L))
    \]
    
    It is therefore immediate to prove that $\lcm$ is in $O(L^3)$.
    \end{proof}
\end{enumerate}
\begin{remark}
The proof on Page 629 shows that $\gcd$ is in $O(L^3)$.
\end{remark}

\bigskip
\begin{exercise}
\end{exercise}
\begin{proof}
We have $\forall x\geq2$:
\[
    -\frac{2}{3x^2}+\int_{x}^{x+1}\frac{1}{y^2}d y=\frac{2x-1}{3x^2(1+x)}\geq0
\] 

Therefore,
\[
    \sum_q\frac{1}{q^2}\leq\sum_{x=2}^{+\infty}\frac{1}{x^2}\leq\frac{3}{2}\int_2^{+\infty}\frac{1}{y^2}d y=\frac{3}{4}
\]
\end{proof}

\subsection{Application: factoring}
\begin{exercise}
\end{exercise}
\begin{enumerate}
    \item \begin{proof}
    The case $a=1$ is trivial; otherwise, if $a>1$, we have
    \[
        b=\log_a N=\frac{\log_2 N}{\log_2 a}\leq\frac{L}{\log_2 a}\leq L
    \]
    \end{proof}
    \item \begin{proof}
    Let $n\leq L$ be the number of digits of precision, we have
    \begin{itemize}
        \item $y=\log_2 N$ is in $O(M(n)n^\frac{1}{2})$ by Taylor series expansion with repeated squaring, which is in $O(L^{\frac{3}{2}+\varepsilon})$ since $M\in O(n^{1+\varepsilon})$ under the Schönhage–Strassen algorithm.
        \item $x=\frac{y}{b}$ is in $O(\log^2(L))$.
        \item $2^x$ is in $O(M(n)n^\frac{1}{2})\subset O(L^{\frac{3}{2}+\varepsilon})$.
        \item $u_1$ and $u_2$ are in $O(L)$.
    \end{itemize} 
    
    Therefore, it takes at most $O(L^2)$ operations.
    \end{proof}
    
    \item \begin{proof}
    Again, we have
    \begin{itemize}
        \item $u_1^b$ and $u_2^b$ are in $O(M(n)n^\frac{1}{2})\subset O(L^{\frac{3}{2}+\varepsilon})$.
        \item Check if either is equal to $N$ is in $O(L)$.
    \end{itemize} 
    
    Hence, it also takes at most $O(L^2)$ operations.
    \end{proof}
    
    \item The algorithm is as follows:
    \begin{itemize}
        \item For each integer $b$ between $1$ and $L$:
        \begin{itemize}
            \item Let $x=\frac{\log_2 N}{b}$, $u_1=\lfloor 2^x\rfloor$ and $u_2=\lceil 2^x\rceil$.
            \item If $u_1^b=N$, return $u_1$ and $b$; if $u_2^b=N$, return $u_2$ and $b$.
        \end{itemize}
        \item If there's no early return, the algorithm fails,
    \end{itemize}
    
    Since the inner loop is executed $L$ times, the algorithm is simply in $O(L^3)$.
\end{enumerate}

\bigskip
\begin{exercise}
\end{exercise}
\begin{proof}
Steps 1 and 2 are passed since $N$ is odd and is not a non-trivial power. 

We have $r=6$ and $x^3\Mod{91}=64$, which is non-trivial. 

We have $\gcd(64-1,91)=7$, which is a non-trivial factor of $91$.
\end{proof}

\bigskip
\begin{exercise}
\end{exercise}
\begin{proof}
Trivial.
\end{proof}
\end{document}