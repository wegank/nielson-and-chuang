\documentclass[../main.tex]{subfiles}
\begin{document}

\setcounter{chapter}{5}
\setcounter{section}{3}
\setcounter{exercise}{19}
\section{General applications of the quantum Fourier transform}
\subsection{Period-finding}

\begin{exercise}
\end{exercise}
Let $n=\frac{N}{r}$, we have
\begin{align*}
    \hat{f}(l)
    &= \frac{1}{\sqrt{N}}\sum_{x=0}^{N-1}e^{-2\pi i l x/N}f(x) \\
    &= \frac{1}{\sqrt{N}}\sum_{k=0}^{n-1}\sum_{x=0}^{r-1}e^{-2\pi i l (k r+x)/N}f(k r + x) \\
    &= \frac{1}{\sqrt{N}}\sum_{x=0}^{r-1}\left[e^{-2\pi i l x/N}f(x)\sum_{k=0}^{n-1}e^{-2\pi i l k/n}\right]
\end{align*}

If $m=\frac{l}{n}$ is an integer, we have
\[
    \hat{f}(l)
    = \frac{1}{\sqrt{N}}\sum_{x=0}^{r-1}\left[e^{-2\pi i l x/N}f(x) \cdot n\right]
    = \frac{\sqrt{N}}{r}\sum_{x=0}^{r-1}e^{-2\pi i m x/r}f(x)
\]

Otherwise, we have 
\[
    \hat{f}(l) = 0
\]
\begin{remark}
The square root should be removed from $\sqrt{\frac{N}{r}}$.
\end{remark}

\bigskip
\begin{exercise}
\end{exercise}
\begin{enumerate}
    \item \begin{proof}
    We have
    \begin{align*}
        U_y\ket{\hat{f}(l)} 
        &= U_y\left[\frac{1}{r}\sum_{x=0}^{r-1}e^{-2\pi i l x/r}\ket{f(x)}\right]\\
        &= \frac{1}{r}\sum_{x=0}^{r-1}e^{-2\pi i l x/r}f(x+y)=e^{2\pi i l y/r}\ket{\hat{f}(l)}
    \end{align*}
    
    So the corresponding eigenvalues are $e^{2\pi i l y/r}$.
    \end{proof}
    
    \item \begin{proof}
    Since we have
    \[ 
        \ket{f(x_0)}=\frac{1}{\sqrt{r}}\sum_{l=0}^{r-1}e^{2\pi i l x_0/r}\ket{\hat{f}(l)}
    \]
    
    According to Exercise 5.8, we have
    \begin{mdframed}
    \begin{enumerate}[label=(\arabic*)]
        \item \quad $\ket{0}\ket{f(x_0)}$ \tabto{4cm} initial state
        \item \quad $\to$ $\widehat{l x_0/r}$ \tabto{4cm} apply quantum phase estimation (with $p\geq\frac{1-\varepsilon}{r}$)
        \item \quad $\to$ $r$ \tabto{4cm} apply continued fractions algorithm
    \end{enumerate}
    \end{mdframed}
    
    So we can also solve the period-finding problem using $U_y$.
    \end{proof}
\end{enumerate}

\subsection{Discrete logarithms}
\begin{exercise}
\end{exercise}
\begin{proof}
We have
\begin{align*}
    \ket{\hat{f}(l_1,l_2)}
    &= \frac{1}{r\sqrt{r}} \sum_{x_1=0}^{r-1}\sum_{x_2=0}^{r-1}e^{-2\pi i(l_1x_1+l_2x_2)/r}\ket{f(x_1,x_2)} \\
    &= \frac{1}{r\sqrt{r}} \sum_{x_1=0}^{r-1}\sum_{x_2=0}^{r-1}e^{-2\pi i(l_1x_1+l_2x_2)/r}\ket{f(0,s x_1+x_2)} \\
    &= \frac{1}{r\sqrt{r}} \sum_{x_1=0}^{r-1}\sum_{j=s x_1}^{s x_1+r-1}e^{-2\pi i(l_1x_1+l_2(j-s x_1))/r}\ket{f(0,j)} \\
    &= \frac{1}{r\sqrt{r}} \sum_{x_1=0}^{r-1}\left[e^{-2\pi i(l_1-l_2 s)x_1/r}\sum_{j=s x_1}^{s x_1+r-1}e^{-2\pi i l_2 j/r}\ket{f(0,j)}\right] \\
    &= \frac{1}{r\sqrt{r}} \left[\sum_{x_1=0}^{r-1}e^{-2\pi i(l_1-l_2 s)x_1/r}\right]\left[\sum_{j=0}^{r-1}e^{-2\pi i l_2 j/r}\ket{f(0,j)}\right]
\end{align*}

If $(l_1-l_2 s)$ is an integer multiple of $r$, we have
\[
    \ket{\hat{f}(l_1,l_2)}=\frac{1}{\sqrt{r}}\sum_{j=0}^{r-1}e^{-2\pi i l_2 j/r}\ket{f(0,j)}
\]

Otherwise, we have
\[
    \ket{\hat{f}(l_1,l_2)}=0
\]
\end{proof}
\begin{remark}
We have slightly modified (5.70) and (5.72).
\end{remark}

\bigskip
\begin{exercise}
\end{exercise}
\begin{proof}
We have
\begin{align*}
    &\quad\frac{1}{\sqrt{r}}\sum_{l_1=0}^{r-1}\sum_{l_2=0}^{r-1}e^{2\pi i(l_1x_1+l_2x_2)/r}\ket{\hat{f}(l_1,l_2)} \\
    &= \frac{1}{\sqrt{r}}\sum_{l_1=0}^{r-1}\sum_{l_2=0}^{r-1}e^{2\pi i(l_1x_1+l_2x_2)/r}\frac{1}{\sqrt{r}}\sum_{j=0}^{r-1}e^{-2\pi i l_2 j/r}\mathds{1}_{r\mathbb{Z}}(l_1-l_2s)\ket{f(0,j)} \\
    &= \frac{1}{r}\sum_{l_1=0}^{r-1}\sum_{l_2=0}^{r-1}\sum_{j=0}^{r-1}e^{2\pi i(l_1x_1+l_2(x_2-j))/r}\mathds{1}_{r\mathbb{Z}}(l_1-l_2s)\ket{f(0,j)} \\
    &= \frac{1}{r}\sum_{l_2=0}^{r-1}\sum_{j=0}^{r-1}e^{2\pi i(l_2(s x_1+x_2-j))/r}\ket{f(0,j)} \\
    &= \frac{1}{r}\sum_{j=0}^{r-1}r\mathds{1}_{r\mathbb{Z}}(s x_1+x_2-j)\ket{f(0,j)} = \ket{f(0,s x_1+x_2)} = \ket{f(x_1,x_2)}
\end{align*}
\end{proof}
\begin{remark}
We have slightly modified (5.73).
\end{remark}

\bigskip
\begin{exercise}
\end{exercise}
Let the continued fraction of $\widehat{s}=\frac{\widehat{s l_2/r}}{\widehat{l_2/r}}$ be $[s_0,s_1,\cdots,s_m]$. We have
\begin{itemize}
    \item If $s_1$ doesn't exist, i.e. $\hat{s}=[s_0]$, we have $s=s_0$.
    \item If $s_1=1$, we have $\hat{s}<s$, so $s=s_0+1$.
    \item Otherwise, we have $\hat{s}>s$, so $s=s_0$.
\end{itemize}

\bigskip
\begin{exercise}
\end{exercise}
We have
\begin{itemize}
    \item $b^{x_1}a^{x_2}$ is in $O(M(t)t)$, according to Box 5.2.
    \item $\ket{y}\ket{b^{x_1}a^{x_2}}\mapsto\ket{y}\ket{y\oplus b^{x_1}a^{x_2}}$ is nothing more than $t$ $\CNOT$s.
\end{itemize}

So we need $O(M(t)t)\subset O(t^{2+\varepsilon})$ gates.

\subsection{The hidden subgroup problem}
\begin{exercise}
\end{exercise}
\begin{proof}
We re-express (5.77) as
\[
    \sum_{h\in K}\prod_{i=1}^M e^{-2\pi i l h_i/\lvert G\rvert}=\lvert K\rvert
\]

where $h_i\in K_{p_i}\leq\mathbb{Z}_{p_i}$.
\end{proof}

\bigskip
\begin{exercise}
\end{exercise}
Let $(a_1,\cdots,a_k)$ be a generating set of $G$. Let
\[
    f:\begin{cases}\mathbb{Z}_q^k\to G\\(x_1\cdots x_k)\mapsto\prod_{i=1}^k a_k x^k\end{cases}
\]

Since $K=\ker f$, we have $\mathbb{Z}_q^k/K\cong G$. The generator of $K$ is given by HSP.

The final step is to calculate a set of generators for $\mathbb{Z}_q^k$, and apply $f$ on it to obtain a set of generators for $G$, which decomposes $G$. \cite{DBLP:journals/corr/cs-DS-0101004}

\bigskip
\begin{exercise}
\end{exercise}
We can apply, as in (5.4.2):

\begin{mdframed}
Runtime: One use of $U$, and $O(L^2)$ operations. Succeeds with probability $O(1)$.
\begin{enumerate}[label=(\arabic*)]
    \item \quad $\ket{0}\ket{0}$ \tabto{6cm} initial state
    \item \quad $\begin{aligned}\to&\frac{1}{\sqrt{\lvert G\rvert}}\sum_{g\in G}\ket{g}\ket{f(g)}\\&=\frac{1}{\lvert G\rvert}\sum_{l=0}^{\lvert G\rvert-1}\sum_{g\in G}e^{2\pi i l g/\lvert G\rvert}\ket{g}\ket{\hat{f}(l)}\end{aligned}$ \tabto{7cm} create superposition and apply $U$
    \item \quad $\to$ $\displaystyle\frac{1}{\sqrt{\lvert G\rvert}}\sum_{l=0}^{\lvert G\rvert-1}\ket{\widehat{l/\lvert G\rvert}}\ket{\hat{f}(l)}$ \tabto{6cm} apply inverse Fourier transform to first register
    \item \quad $\to$ $\widehat{l/\lvert G\rvert}$ \tabto{6cm} measure first register
    \item \quad $\to$ $l$ \tabto{6cm} apply continued fractions algorithm
\end{enumerate}
\end{mdframed}

To determine $l$, then determine elements of $K$ and its generating set classically.

\bigskip
\begin{exercise}
\end{exercise}
Trivial.

\subsection{Other quantum algorithms?}

\end{document}