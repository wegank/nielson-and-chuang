\documentclass{report}
\usepackage[utf8]{inputenc}
\usepackage[T1]{fontenc}
\usepackage[a4paper]{geometry}
\usepackage{biblatex}
\addbibresource{res.bib}
\usepackage{authblk}
\usepackage{amsmath}
\usepackage{amsfonts}
\usepackage{amssymb}
\usepackage{amsthm}
\usepackage{dsfont}
\usepackage{stmaryrd}
\usepackage{graphicx}
\usepackage{float} 
\usepackage{subfigure}
\usepackage{braket}
\usepackage{qcircuit}

\usepackage{tabto}
\usepackage{enumitem}
\usepackage[nobreak=true]{mdframed}

\usepackage{subfiles}


\theoremstyle{definition}
\newtheorem{exercise}{Exercise}[chapter]
\newtheorem{problem}{Problem}[chapter]
\theoremstyle{remark}
\newtheorem*{remark}{Remark}

\DeclareMathOperator{\lcm}{lcm}
\newcommand{\tr}{\mathrm{tr}}
\newcommand{\Diag}{\mathrm{Diag}}
\newcommand{\Sch}{\mathrm{Sch}}
\newcommand{\NOT}{\textsc{NOT}}
\newcommand{\CNOT}{\textsc{CNOT}}
\newcommand{\Mod}[1]{\ (\mathrm{mod}\ #1)}


\title{QCQI Worksheet}
\author[1]{Weijia Wang}
\affil[1]{Department of Mathematics, Sorbonne University}

\begin{document}
\maketitle
\tableofcontents
\newpage

\setcounter{chapter}{1}
\chapter{Introduction to quantum mechanics}

\subfile{week04/2.4}

\newpage
\subfile{week05/2.5}
\newpage
\subfile{week05/2.6}

\setcounter{chapter}{3}
\chapter{Quantum circuits}

\subfile{week05/4.2}

\newpage
\subfile{week06/4.3}
\newpage
\subfile{week06/4.4}

\chapter{The quantum Fourier transform and its applications}
\subfile{week09/5.1}
\newpage
\subfile{week09/5.2}

\newpage
\subfile{week10/5.3}
\newpage
\subfile{week10/5.4}
\printbibliography

\end{document}
