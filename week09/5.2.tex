\documentclass[../main.tex]{subfiles}
\begin{document}

\setcounter{chapter}{5}
\setcounter{section}{1}
\setcounter{exercise}{6}
\section{Phase estimation}

\begin{exercise}
\end{exercise}
\begin{proof}
It is easy to see that
\[
    \ket{j}\ket{u}\mapsto\ket{j}\prod_{i=1}^t U^{2^{t-i} j_i}\ket{u}=\ket{j} U^{\sum_{i=1}^t 2^{t-i}j_i}\ket{u}=\ket{j}U^j\ket{u}
\]
\end{proof}

\bigskip
\begin{exercise}
\end{exercise}
We see that
\begin{itemize}
    \item The output $\sum_u c_u\ket{\widetilde{\varphi_u}}\ket{u}$ being in $\ket{\widetilde{\varphi_u}}\ket{u}$ is of probability $\lvert c_u\rvert^2$.
    \item Getting $\varphi_u$ accurate to $n$ bits in the first condition is of probability at least $1-\epsilon$.
\end{itemize}

So the probability for measuring $\varphi_u$ accurate to $n$ bits is at least $\lvert c_u\rvert^2(1-\epsilon)$.

\bigskip
\begin{exercise}
\end{exercise}
We have $\varphi=0=0.0$ or $\varphi=\frac{1}{2}=0.1$, so a $1$-bit approximation suffices.

From Figure 5.1 we have
\[
    FT^\dagger=H^\dagger=H
\]

According to Figure 5.3, the transformation is given by
\[
    \begin{array}{c}\Qcircuit @C=1em @R=.7em {
    \lstick{\ket{0}} & \gate{H} & \ctrl{1} & \gate{H} & \meter \\
    \lstick{\ket{u}} & \qw & \gate{U} & \qw & \rstick{\ket{u^\prime}} \qw
    }\end{array}
\]
which is exactly the same as in Exercise 4.34.

If the measurement gives $0$ (resp. $1$), $\ket{u^\prime}$ is in the eigenspace associated with $1$ (resp. $-1$).


\end{document}