\documentclass[../main.tex]{subfiles}
\begin{document}

\setcounter{chapter}{2}
\setcounter{section}{5}
\section{EPR and the Bell inequality}

\begin{problem}[Functions of the Pauli matrices]
\end{problem}

\begin{proof}
We prove the case where $f$ is analytic. We've seen in Exercise 2.60 that
\begin{equation*}
    \theta\Vec{n}\cdot\Vec{\sigma}=\theta\cdot\frac{I+\Vec{n}\cdot\Vec{\sigma}}{2}+(-\theta)\cdot\frac{I-\Vec{n}\cdot\Vec{\sigma}}{2}
\end{equation*}

So that $\frac{I\pm\Vec{n}\cdot\Vec{\sigma}}{2}$ are the Forbenius covariants of their correspondent eigenvalues.

Then, by Sylvester's formula, we have
\begin{align*}
    f(\theta\Vec{n}\cdot\Vec{\sigma})
    &= f(\theta)\cdot\frac{I+\Vec{n}\cdot\Vec{\sigma}}{2}+f(-\theta)\cdot\frac{I-\Vec{n}\cdot\Vec{\sigma}}{2} \\
    &= \frac{f(\theta)+f(-\theta)}{2}I+\frac{f(\theta)-f(-\theta)}{2}\Vec{n}\cdot\Vec{\sigma}
\end{align*}
\end{proof}

\begin{remark}
If $f$ is not analytic, the equality may not be satisfied.
\end{remark}

\bigskip
\begin{problem}[Properties of the Schmidt number]
\end{problem}
\begin{enumerate}
    \item
    \begin{proof}
    We note the Schmidt decomposition of $\ket{\psi}$ as
    \begin{equation*}
        \ket{\psi}=\sum_i\lambda_i\ket{i_A}\ket{i_B}
    \end{equation*}
    
    Then it is easy to see that
    \begin{equation*}
        \rho_A=\tr_B(\ket{\psi}\bra{\psi})=\sum_i\lambda_i^2\ket{i_A}\bra{i_A}
    \end{equation*}
    with the latter being a spectral decomposition.
    
    Thus, we have $\Sch(\psi)=\mathrm{card}((\lambda_i)_i)=\mathrm{card}((\lambda_i^2)_i)=\mathrm{rank}(\rho_A)$.
    \end{proof}
    
    \item
    \begin{proof}
    Similarly, we have
    \begin{equation*}
        \rho_A=\sum_{i,j}\braket{\beta_j\lvert\beta_i}\ket{\alpha_i}\bra{\alpha_j}
    \end{equation*}
    Let $u$ be a matrix where $u_{i j}=\braket{\beta_j\lvert\beta_i}$ (and $\mathrm{rank}(u)=\mathrm{rank}(\rho_A)$). The number of terms in the decomposition is simply the number of non-zero elements in $u$, which is evidently greater than or equal to its rank, i.e. $\Sch(\psi)$.
    \end{proof}
    
    \item
    \begin{proof}
    We only need to prove the original triangle inequality, i.e.
    \begin{equation*}
        \Sch(\psi)\leq\Sch(\varphi)+\Sch(\gamma)
    \end{equation*}
    
    So that we can end the proof using the fact that
    \begin{equation*}
        \ket{\varphi}=\frac{1}{\alpha}\ket{\psi}+(-\frac{\beta}{\alpha})\ket{\gamma}
    \end{equation*}
    \begin{equation*}
        \ket{\gamma}=\frac{1}{\beta}\ket{\psi}+(-\frac{\alpha}{\beta})\ket{\varphi}
    \end{equation*}
    
    To prove the triangle inequality, it is sufficient to see that, if we express $\ket{\varphi}$ and $\ket{\gamma}$ as
    \begin{equation*}
        \ket{\varphi}=\sum_i \lambda_i\ket{\alpha_i}\ket{\beta_i}
    \end{equation*}
    \begin{equation*}
        \ket{\gamma}=\sum_j \lambda^\prime_j\ket{\alpha^\prime_j}\ket{\beta^\prime_j}
    \end{equation*}
    
    Then $\rho^A$ can be expressed as
    \begin{equation*}
        \rho^A=\sum_i \ket{\tau_i}\bra{\alpha_i}+\sum_j \ket{\tau^\prime_j}\bra{\alpha^\prime_j}
    \end{equation*}
    
    Therefore,
    \begin{equation*}
        \Sch(\psi)=\mathrm{rank}(\rho^A)\leq\mathrm{card}\left(\left(\bra{\alpha_i}\right)_i\right)+\mathrm{card}\left(\left(\bra{\alpha^\prime_j}\right)_j\right)=\Sch(\varphi)+\Sch(\gamma)
    \end{equation*}
    \end{proof}
\end{enumerate}

\bigskip
\begin{problem}
\end{problem}
\begin{enumerate}
    \item
    \begin{proof}
    We see that
    \begin{align*}
        &\ (Q\otimes S+R\otimes S+R\otimes T-Q\otimes T)^2 \\
        &= ((Q+R)\otimes S+(R-Q)\otimes T)^2 \\ 
        &= (Q+R)^2\otimes S^2+(R-Q)^2\otimes T^2 + (Q+R)(R-Q)\otimes ST + (R-Q)(Q+R)\otimes TS \\
        &= (2I+\{Q,R\})\otimes I+(2I-\{Q,R\})\otimes I+[Q,R]\otimes ST+[R,Q]\otimes TS \\
        &= 4I + [Q,R]\otimes [S,T]
    \end{align*}
    \end{proof}
    
    \item
    \begin{proof}
    We have
    \begin{align*}
        &\ \braket{Q\times S}+\braket{R\times S}+\braket{R\times T}-\braket{Q\times T} \\
        &= \braket{Q\otimes S+R\otimes S+R\otimes T-Q\otimes T} \\
        &\leq \sqrt{\braket{Q\otimes S+R\otimes S+R\otimes T-Q\otimes T}^2} \\
        &= \sqrt{\braket{4I + [Q,R]\otimes [S,T]}} \\
        &= \sqrt{4 + \braket{[Q,R]}\braket{[S,T]}} \\
        &\leq \sqrt{4 + 2\times 2} = 2\sqrt{2}
    \end{align*}
    
    where we applied
    \begin{align*}
        \left\lvert\braket{[Q,R]}\right\rvert
        &= \lvert\braket{Q R} - \braket{R Q}\rvert \\
        &\leq \lvert\braket{Q R}\rvert + \lvert\braket{R Q}\rvert \\
        &\leq 2\sqrt{\braket{Q^2}\braket{R^2}} = 2\lvert\braket{I}\rvert = 2
    \end{align*}
    \end{proof}
\end{enumerate}
\end{document}