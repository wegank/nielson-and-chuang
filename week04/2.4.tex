\documentclass[../main.tex]{subfiles}
\begin{document}

\setcounter{chapter}{2}
\setcounter{section}{3}
\setcounter{exercise}{70}
\section{The density operator}

\setcounter{subsection}{1}
\subsection{General properties of the density operator}

\begin{exercise}[Criterion to decide if a state is mixed or pure]
\end{exercise}

\begin{proof}
    We have
    \begin{align*}
        \tr\left(\rho^2\right)
        &= \tr\left(\sum_i p_i\ket{\psi_i}\bra{\psi_i}\right)^2 \\
        &= \tr\left(\sum_{i,j}p_i p_j \braket{\psi_i\vert\psi_j} \ket{\psi_i}\bra{\psi_j}\right) \\
        &= \sum_{i,j} p_i p_j \braket{\psi_i\vert\psi_j} \tr\left(\ket{\psi_i}\bra{\psi_j}\right) \\
        &= \sum_{i,j} p_i p_j \braket{\psi_i\vert\psi_j}^2 \\
        &\leq \sum_{i,j} p_i p_j \braket{\psi_i\vert\psi_i} \braket{\psi_j\vert\psi_j} = \sum_{i,j} p_i p_j = 1
    \end{align*}
    
    Additionally, for all $i\neq j$, we have $\braket{\phi_i\vert\phi_j}^2<\braket{\psi_i\vert\psi_i} \braket{\psi_j\vert\psi_j}$.
    
    Thus, the equality holds iff only one $\ket{\psi_i}$ has a non-zero probability, i.e. $\rho$ is a pure state.
\end{proof}

\begin{remark}
    In the unofficial solution manual, the author constructs another ensemble of pure states using the spectral decomposition, so that $\braket{\psi_i\vert\psi_j}=\delta_{i j}$.
\end{remark}

\bigskip
\begin{exercise}[Bloch sphere for mixed states]
\end{exercise}
\begin{enumerate}
    \item
    \begin{proof}
        Using the fact that $\{I,\sigma_1,\sigma_2,\sigma_3\}$ forms a basis of $\mathrm{M}_{2,2}(\mathbb{C})$, one can write
        \begin{equation*}
            \rho = \frac{c I + r_1 \sigma_1 + r_2 \sigma_2 + r_3 \sigma_3}{2}=\frac{c I + \vec{r}\cdot\vec{\sigma}}{2}
        \end{equation*}
        
        Since $(\sigma_i)_i$ are with trace zero, we have
        \begin{equation*}
            \tr(\rho)=\frac{c\ \tr(I)}{2}=c \implies c=1
        \end{equation*}
        
        Similarly,
        \begin{equation*}
            \tr(\rho^2)=\tr(\frac{I+2\ \vec{r}\cdot\vec{\sigma}+\lvert \vec{r}\rvert^2 I}{4})=\frac{1+\lvert \vec{r}\rvert^2}{2} \implies \lvert \vec{r}\rvert\leq 1
        \end{equation*}
    \end{proof}
    
    \item
    We have $\vec{r}=0$, since $(\sigma_i)_i$ are linearly independent.
    
    \item
    \begin{proof}
        $\rho$ is a pure state if and only if $\tr(\rho^2)=1$, i.e. $\lvert\vec{r}\rvert=1$.
    \end{proof}
    
    \item
    \begin{proof}
        We simply have
        \begin{align*}
            \ket{\psi}\bra{\psi}
            &= \cos^2\frac{\theta}{2}\ket{0}\bra{0}+\sin^2\frac{\theta}{2}\ket{1}\bra{1}+\cos\frac{\theta}{2}\sin\frac{\theta}{2}\left(e^{-i\varphi}\ket{0}\bra{1}+e^{i\varphi}\ket{1}\bra{0}\right) \\
            &= \frac{1}{2}(I+\cos\theta \ \sigma_3)+\frac{\sin\theta}{2}(\cos\varphi\ \sigma_1+\sin\varphi\ \sigma_2) \\
            &= \frac{I+(\cos\varphi\sin\theta, \sin\varphi\sin\theta, \cos\theta)\cdot\vec{\sigma}}{2} = \frac{I+\vec{r}\cdot\vec{\sigma}}{2}
        \end{align*}
        We see that $\vec{r}$ coincides indeed with the vector in the Bloch sphere.
    \end{proof}
\end{enumerate}

\bigskip
\begin{exercise}
\end{exercise}
\begin{proof}
    It is obvious that a Gram-Schmidt process with $\ket{\psi}$ as the initial vector builds a minimal ensemble for $\rho$. To prove the equality for every minimal ensemble $\{p_i,\ket{\psi_i}\}$ (not necessarily orthonormal!), we note the spectral decomposition of $\rho$ on its support as
    \begin{equation*}
        \rho=\sum_k \lambda_k\ket{k}\bra{k}
    \end{equation*}
    
    So that
    \begin{equation*}
        \rho^{-1}=\sum_k\frac{1}{\lambda_k}\ket{k}\bra{k}
    \end{equation*}
    
    Since the two ensembles generate both $\rho$, there exists a unitary matrix $u$ such that
    \begin{equation*}
        \ket{\tilde{\psi_i}}=\sum_k u_{i k}\ket{\tilde{k}}\iff \ket{\psi_i}=\sum_k u_{i k}\sqrt{\frac{\lambda_k}{p_i}}\ket{k}
    \end{equation*}
    
    Therefore, we have
    \begin{equation*}
        \frac{1}{\braket{\psi_i\vert\rho^{-1}\vert\psi_i}}
        = \frac{1}{\sum_k\frac{1}{\lambda_k}\braket{\psi_i\vert k}^2}=\frac{1}{\frac{1}{p_i} \sum_k u_{i k}^2}=\frac{1}{\frac{1}{p_i}}=p_i
    \end{equation*}
\end{proof}

\subsection{The reduced density operator}
\begin{exercise}
\end{exercise}
\begin{proof}
    We have
    \begin{equation*}
        \rho^A = \tr_B\left(\ket{a}\ket{b}\bra{b}\bra{a}\right) = \ket{a}\bra{a}\tr_B\left(\ket{b}\bra{b}\right) = \ket{a}\bra{a}
    \end{equation*}
    
    So $\rho^A$ is a pure state.
\end{proof}

\bigskip
\begin{exercise}
\end{exercise}
One should expect that $\rho=\frac{I}{2}$ for each qubit in each of the four Bell states.

\end{document}