\documentclass[../main.tex]{subfiles}
\begin{document}

\setcounter{chapter}{4}
\setcounter{section}{3}
\setcounter{exercise}{31}
\section{Measurement}

\begin{exercise}
\end{exercise}
\begin{proof}
It seems to me that $P_m$ means $I\otimes\ket{m}\bra{m}$ in the formula to prove, so I'll simply mix the two notations without ambiguity. We have
\[
    \rho^\prime=\sum_m P_m\rho P_m^\dagger=\sum_m P_m\rho P_m
\]

and
\[
    \tr_2(\rho^\prime)=\tr_2\left(\sum_m P_m\rho P_m\right)=\sum_m\rho_1\tr\left(\rho_2 P_m^2\right)=\rho_1\tr\left(\rho_2\sum_{m}P_m\right)=\rho_1\tr\left(\rho_2\right)=\tr_2(\rho)
\]
\end{proof}

\bigskip
\begin{exercise}[Measurement in the Bell basis]
\end{exercise}
We have
\[
    A = \begin{array}{c}\Qcircuit @C=1em @R=.7em {
    & \ctrl{1} & \gate{H} & \qw \\
    & \targ & \qw & \qw \\
    }\end{array}
    = (H\otimes I) \Diag(I,X)
    = \frac{1}{\sqrt{2}}\begin{bmatrix}I&X\\I&-X\end{bmatrix}
    = \frac{1}{\sqrt{2}}\begin{bmatrix}1&0&0&1\\0&1&1&0\\1&0&0&-1\\0&1&-1&0\end{bmatrix}
\]

So that it performs a measurement in the basis of the Bell states.

For the second question, let $M_i$ be the matrix satisfying
\[
    (M_i)_{j k}=\begin{cases}A_{j k} & i = j\\0&i\neq j\end{cases}
\]

It is easy to check that $(M_i)_i$ correspond to the desired measurement operators, i.e.,

\[
    \{M_i^\dagger M_i\}_i = \{\ket{\beta_{j k}}\bra{\beta_{j k}}\}_{j k}
\]

\bigskip
\begin{exercise}[Measuring an operator]
\end{exercise}
\begin{proof}
We see that
\[
    \begin{array}{c}\Qcircuit @C=1em @R=.7em {
    \lstick{\ket{0}} & \gate{H} & \ctrl{1} & \gate{H} & \qw \\
    \lstick{\ket{\psi_{\mathrm{in}}}} & \qw & \gate{U} & \qw & \qw
    }\end{array}
    = \frac{1}{2}\left(\ket{0}\left((1+U)\ket{\psi_\mathrm{in}}\right)+\ket{1}\left((1-U)\ket{\psi_\mathrm{in}}\right)\right)
\]

As
\[
    U((1+U)\ket{\psi_\mathrm{in}}) = 1\cdot((1+U)\ket{\psi_\mathrm{in}})
\]
\[
    U((1-U)\ket{\psi_\mathrm{in}}) = -1\cdot((1-U)\ket{\psi_\mathrm{in}})
\]

We find that $\ket{\psi_{\mathrm{out}}}$ is indeed an eigenvector.
\end{proof}

\bigskip
\begin{exercise}[Measurement commutes with controls]
\end{exercise}
\begin{proof}
We have
\begin{align*}
    \begin{array}{c}\Qcircuit @C=1em @R=.7em {
    \lstick{\alpha\ket{0}+\beta\ket{1}} & \meter & \cctrl{1} & \\
    \lstick{\ket{\psi}} & \qw & \gate{U} & \qw
    }\end{array} 
    = \alpha\ket{\psi} + \beta U \ket{\psi} 
    = \begin{array}{c}\Qcircuit @C=1em @R=.7em {
    & \ctrl{1} & \meter \\
    & \gate{U} & \qw & \qw
    }\end{array}
\end{align*}

So that we can defer the measurement.
\end{proof}
\end{document}